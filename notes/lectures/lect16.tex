\section{Vertex Coloring}

\begin{definition}[\(k\)-coloring and \(k\)-chromatic]
  A coloring that uses \(k\) colors is called a \(k\)-coloring.
  If \(\chi(G)=k\), then \(G\) is said to be \(k\)-chromatic.
\end{definition}

If \(G\) is a \(k\)-chromatic graph, then we can partition 
\(V(G)\) into \(k\) independent sets called \textit{color
classes}: \(V_1, V_2, \ldots, V_k\). However, we cannot do the
same with \(k-1\) independent sets. Note that
\[ \sum_{i=1}^k V_k = n \]

\begin{theorem}
  A graph has chromatic number 2 if and only if \(G\) is a
  non-trivial bipartite graph.
\end{theorem}

\begin{proof}
  To show the forward direction, assume that \(G\) has chromatic
  number 2. As a result, we have 2 independent sets according to
  the coloring. Then, let \(A\) be the set of vertices colored
  red and \(B\) be for blue. Then, \(A\) and \(B\) form the
  bipartite sets that make up a bipartite graph.

  To show the inverse direction, we can simply color the two
  partite sets of the graph different colors.
\end{proof}

\begin{theorem}
  Every graph \(G\) with \(m\) edges satisfies
  \[ \chi(G) \leq \frac{1}{2}+\sqrt{2 m+\frac{1}{4}} \]
\end{theorem}

\begin{proof}
  Let \(k = \chi(G)\). Then, \(G\) has least one edge between two
  color classes. If not, then we can join the color classes that
  share no edge with each other since they are independent.
  Thus, we have that
  \[ m \geq \binom{k}{2} = \frac{k(k-1)}{2} \]
  Solving this for \(k\) gives the bound in the claim.
\end{proof}

\begin{definition}
  The \textit{vertex independence number} of \(G\), denoted
  \(\alpha(G)\), is the maximum number of vertices of \(G\) such
  that no two of which are adjacent.
\end{definition}

\begin{theorem}
  \[ \frac{n}{\alpha(G)} \leq \chi(G) \leq n - \alpha(G) + 1 \]
\end{theorem}

\begin{proof}
  First, we will show the lower bound. Recall that we have
  \[
    \begin{aligned}
      |V_1| + |V_2| + \cdots + |V_{\chi(G)}| &= n \\
      \alpha(G) \cdot \chi(G) &\geq n \\
      \chi(G) &\geq \frac{n}{\alpha(G)}
    \end{aligned}
  \]

  For the upper bound, we already know that 
  \[ |V_1| + |V_2| + \cdots + |V_{\chi(G)}| = n \]
  which is 
  \[ \alpha(G) + 1 + \cdots + 1 = n \]
  in the case that we need the most colors.
  We can color the biggest independent set with color 1 and color
  the rest with different colors. Then, we have
  \[ \alpha(G) + \underbrace{1+1 + \cdots + 1}_{\chi(G)-1} = n \]
  Finally, we would have
  \[
    \begin{aligned}
      \chi(G) &= n + 1 - \alpha(G) \\
      \chi
    \end{aligned}
  \]
\end{proof}

