\begin{theorem}[Menger: Set Version]
  Let \(G = (V, E)\) be a graph and let \(A, B \subseteq V\).
  Then, the minimum number of vertices separating \(A\) and \(B\)
  in \(G\) is equal to the maximum number of disjoint \(A-B\)
  paths in \(G\).
\end{theorem}

\begin{proof}
  We reduce this version to the original version by attaching
  \(u\) to \(A\) and \(v\) to \(B\). Then, claim that the minimum
  cut-vertex is the same as the maximum disjoint paths. Once we
  remove \(u, v\), the claim of this set version still satsifies.
\end{proof}

\begin{theorem}[Menger's Theorem: Global]
  \phantom{os}

  \begin{enumerate}[label=(\roman*.)]
    \item A nontrivial graph \(G\) is \(k\)-connected for some
      integer \(k \geq 2\) if and only if for each pair \(u, v\)
      of distinct vertices of \(G\) there are at least \(k\)
      internally disjoint \(u-v\) paths in \(G\).
    \item A nontrivial graph \(G\) is \(k\)-edge-connected if and
      only if \(G\) contains \(k\) pairwise edge-disjoint \(u-v\)
      paths for each pair \(u, v\) of distinct vertices of \(G\).
  \end{enumerate}
\end{theorem}

\begin{remark}
  The difference of this version and the original version is that
  \(u, v\) can be adjacent in the first case.
\end{remark}

\begin{corollary}
  If \(G\) is a \(k\)-connected graph and \(u, v_1, v_2, \ldots,
  v_k\) are \(k+1\) distinct vertices of \(G\), then there exist
  internally disjoint \(u-v_i\) paths (\(1 \leq i \leq k\)) in
  \(G\).
\end{corollary}

\begin{proof}
  We construct a new graph \(H\) by adding a new vertex \(w\) and
  joining \(w\) to the \(v_i\)'s. Since \(G\) is \(k\)-connected,
  you need to remove at least \(k\) vertices to disconnect the 
  graph. By Menger's Theorem: \textit{Global Edition}, there are
  \(k\) internally disjoint \(u-w\) paths in \(H\). Thus, there
  are \(k\) disjoint \(u-v_i\) paths in \(G\).
\end{proof}

\begin{theorem}
  If \(G\) is a \(k\)-connected graph with \(k\geq2\), then every
  \(k\) vertices of \(G\) lie on a common cycle of \(G\).
\end{theorem}

% \begin{proof}[Proof (Sketch)]
%   The main idea of the proof is to assume that if there is a
%   smaller cycle with \(l<k\) vertices, then you can always
%   include more vertices into it to make it bigger.
% \end{proof}

\begin{proof}
  Let \(G\) be a \(k\)-connected graph and let \(S = \{v_1, v_2,
  \ldots, v_k\}\) be the set of vertices that does \textit{not}
  lie on any common cycle in \(G\). Now, let \(C\)
  be a cycle in \(G\) that contains as many elements in \(S\) as
  possible. Suppose to contradict that \(|C| = l < k\).

  Let \(v^* \in S\) be not in \(C\). Then, since \(G\) is
  \(k\)-connected, we can apply Menger's Theorem:
  \textit{Global Edition} and find \(k\) disjoint paths from
  some \(v_i\)'s that are already included in \(C\). This is a
  contradiction since we supposed that \(C\) already includes as
  many vertices of \(S\) as possible.

  Therefore, we can conclude that if \(G\) is a \(k\)-connected
  graph with \(k \geq 2\), then every \(k\) vertices of \(G\)
  must lie on a common cycle of \(G\).
\end{proof}

