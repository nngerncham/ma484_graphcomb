\chapter{Planarity}

\section{Core Definitions and Theorems}

\begin{definition}
  A graph \(G\) is called a planar graph if \(G\) can be drawn in
  the plane so that no two of its edges cross each other.
\end{definition}

\begin{definition}
  A graph \(G\) is called a plane graph if it is drawn in the
  plane so that no two edges of \(G\) cross.
\end{definition}

\begin{nexample}
  We can easily show that \(K_{4}\) is a planar graph.
\end{nexample}

\begin{nexample}
  Did you know? The world map can be turned into a planar graph!
  
  Firstly, we can construct the world map as a graph where we
  assign a vertex to each country. Then, if two countries share a
  border, then you can draw an edge to it. Hence, we can just
  draw the edges through the border and get a planar graph.

  However, this does not mean that we can turn every map into a
  planar graph. Still, it shows that we can make \textit{a}
  planar graph from \textit{a} world map.
\end{nexample}

\begin{theorem}[The Euler Identity]
  If \(G\) is a connected plane graph of order \(V\), size \(E\)
  and having \(F\) regions (including outer region), then
  \[ V-E+F=2 \]
\end{theorem}

\begin{proof}
  We will show this by case.
  \begin{enumerate}
    \item If \(G\) is a tree, then \(F=1\) and \(E=V-1\). It
      follows that
      \[
        \begin{aligned}
          V-E+F &= V-(V-1)+1 \\
                &= 2
        \end{aligned}
      \]

    \item Otherwise, \(G\) is not a tree. Then, \(G\) must
      contain a cycle.
      We will apply induction on the number of edges.
      Let \(e\) be an edge of the cycle.
      Consider \(G' = G-e\). Notice that the \(V_{G'}\) remains
      the same while \(E_{G'}\) and \(F_{G'}\) goes down by one
      each. It follows that for \(G\),
      \[
        \begin{aligned}
          V - E + F &= V - E + 1 + F - 1 \\
                    &= \underbrace{V-(E-1)+(F-1)}_\text{I.H.} \\
                    &= 2 
        \end{aligned}
      \]
      Therefore, the claim holds.
  \end{enumerate}
\end{proof}

\begin{theorem}
  If \(G\) is a planar graph of order \(n \geq 3\) and size
  \(m\), then
  \[ m \leq 3 n-6 \]
\end{theorem}

\begin{proof}
  We will show this by case. 
  \begin{enumerate}
    \item \(G\) is connected. Notice that each face
      is surrounded by \(\geq 3\) edges. Let \(M\) be the 
      sum of the number of edges that each face is adjacent to. 
      Then, we would have that 
      \[ 3F \leq M \]
      Notice also that each edge can only be adjacent to at most
      2 faces since the graph is planar. It follows that
      \[ M \leq 2m \]
      Combining these, we have
      \[ F \leq \frac{2m}{3} \]
      Plugging this into The Euler Identity, we have
      \[
        \begin{aligned}
          n - m + F &= 2 \\
          F &= 2 + m - n \\
          \frac{2m}{3} &\geq 2 + m - n \\
          2m &\geq 6 + 3m - 3n \\ 
          3n - 6 &\geq m
        \end{aligned}
      \]

    \item Otherwise, \(G\) is disconnected. Then, you add edges
      until \(G\) is connected and denote the new graph as
      \(G'\). Notice that \(m' > m\) but \(n' = n\). Now that
      \(G'\) is connected, we have that \(m' \leq 3n-6\).
      Therefore, we have that
      \[ m < m' \leq 3n-6 \]
  \end{enumerate}
\end{proof}

\begin{corollary}
  Every planar graph contains a vertex of degree 5 or less.
\end{corollary}

\begin{proof}
  Assume that \(\deg v \geq 6\) for all \(v \in V(G)\). Then, we
  have that
 \[
    \begin{aligned}
      2m &= \sum_{v \in V(G)} \deg v \\
      2m &\geq 6n \\
      m &\geq 3n
    \end{aligned}
  \]
  This is a contradiction since we would otherwise have
  \[ 3n \leq m \leq 3n - 6 \]
  Therefore, every planar graph must contain a vertex of degree
  5 or less.
\end{proof}

