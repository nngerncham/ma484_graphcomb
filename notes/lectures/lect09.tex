  Now, we will show \ref{eqn:gallai-proof-2}. Let \(X\) be the minimum edge 
  cover of \(G\). Then, we have \(|X|=\beta'(G)=l\). Let \(F = G[X]\) or a 
  subgraph induced by \(X\). Notice that \(F\) has no
  trail of length 3. Otherwise, there must be a middle edge \(e\) but \(X
  \setminus \{e\}\) would still be an edge cover of \(G\), creating a
  contradiction. Hence, \(F\) contains no cycle and no paths of length 3 or
  grater. This implies that every component of \(F\) is a star. 

  Notice that each component with \(n_i\) vertices has \(n_i - 1\) edges. So,
  if \(F\) has \(c\) components, then it has \(n-c\) edges and all of these must
  be in the edge cover. Hence, we have that
  \[
  \begin{aligned}
    l \geq n-c \iff c \geq n-l
  \end{aligned}
  \]
  To find \(\alpha'(G)\), we can pick one edge from each component of \(F\).
  Then, it follows that 
  \[ \alpha'(G) = c \geq n-l \]
  This implies that 
  \[ \alpha'(G) + \beta'(G) \geq (n-l) + l = n \] 
\end{proof}

\begin{theorem}
  For every graph \(G\) of order \(n\) containing no isolated vertices, we have
  \[ \alpha(G) + \beta(G) = n \]
\end{theorem}

\begin{proof}
  The proof is similar to Gallai's Theorem and will be omitted.
\end{proof}

\begin{nexample}
  Prove that if \(G\) is a graph of order \(n\) with maximum degree \(\Delta\)
  and has no isolated vertices, then 
  \[ \beta(G) \geq \frac{n}{\Delta + 1} \]

  \begin{proof}
    Let \(G\) be a graph of order \(n\) and suppose that \(G\) has \(k\)
    components. We make the following observations:
    \begin{enumerate}
      \item \(|E(G)| \leq \beta(G) \cdot \Delta\). This is because
        \(\beta(G)\) gives the number of vertices needed to cover every edge,
        and each vertex has at most \(\Delta\) edges incident to it.
      \item \(n-k \leq |E(G)|\). Notice that \(n-k\) is the minimum number of
        edges required for a graph with \(k\) components to have no isolated
        vertices.
      \item \(k \leq \beta(G)\). Otherwise, there would be isolated vertices
        with violates our assumption.
    \end{enumerate}
    Combining all of these together, we have that
    \[ n-\beta(G) \leq n-k \leq |E(G)| \leq \beta(G) \cdot \Delta \]
    Finally, we have that 
    \[
    \begin{aligned}
      n - \beta(G) &\leq \beta(G) \cdot \Delta \\ 
      n &\leq \beta(G) (\Delta + 1) \\ 
      \frac{n}{\Delta + 1} &\leq \beta(G)
    \end{aligned}
    \]
    which is the inequality we wanted to show.
  \end{proof}
\end{nexample}

\begin{remark}
  The bound in the example is weak when \(G\) has many edges. 
\end{remark}
