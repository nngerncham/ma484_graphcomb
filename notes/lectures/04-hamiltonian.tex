\begin{nexample}
  What is the smallest number of edges that will ensure that a simple graph on
  10 vertices is Hamiltonian.
\end{nexample}

\subsection{Traveling Salesman Problem (TSP)}

TSP is a question asking, given \(D\) cities, is there a Hamiltonian Cycle 
that visits all \(n\) cities (vertices) then returns to its starting point
and has total length \(\leq D\). TSP is also an example of an NP-complete 
problem. That is, if such a tour exists, then we are able to verify this within
polynomial time.

The naive solution for \(n\) cities is to go through the \(n!\) different
possible paths and look for the shortest one. \(n!\) can be approximated as
follows using Stirling's Formula.
\[ n! \approx \left(\frac{n}{e}\right)^n \cdot \sqrt{2 \pi n} \]
Asymptotically, this search would take \textit{super-exponential} time:
\(\mathcal{O}(n^n)\). Similarly, searching for a Hamiltonian Cycle on a simple 
graph can be stated as a TSP problem. As such, they would have the same 
running time.
\textbf{not} Hamiltonian 

On the other hand, determining if there exists an Euler Circuit in a graph is a
very simple problem since it only requires linear time to check if the degree of
every vertex is even or not. Another method of determining this is to check if
the graph has a bridge or not. An analogous version of this condition is the cut
vertex---a vertex that disconnects the graph once removed. If a graph has a cut
vertex, then the graph may be Eulerian. However, said graph cannot be
Hamiltonian.
