\begin{theorem}[Cayley's Theorem, 1889]
  There are \(n^{n-2}\) distinct labelled trees on \(n\) vertices.
\end{theorem}

\begin{proof}[Proof with Pr\"{u}fer's Code]
  We will show that there is a bijection \(f : X \to Y\) where \(X\) is the set
  labelled tree of \(n\) vertices and \(Y\) is the \(n-2\) tuples of numbers 1 
  to \(n\). is the set of Prufer's codes. Namely, we will show that
  \(f(X) \subseteq Y\). Then, we will show that \(f^{-1}(Y) \subseteq X\).

  [Left as exercise]
\end{proof}

\chapter{Matching and Covering}

In this chapter, the main question is as follows: Given a graph \(G\), how do we
find the matching \(M\) with maximum size?

\section{Core Definitions}

\begin{definition}[Edge Independent]
  A set \(F\) of edges in a graph is \textit{independent} if no two edges in
  \(F\) are adjacent.
\end{definition}

\begin{definition}[Matching]
  A \textit{matching} \(M\) of graph \(G\) is a subgraph of \(G\) such that
  every edge shares no vertex with any other edge. That is, each vertex in
  matching \(M\) has degree one.
\end{definition}

\begin{definition}[Complete/Perfect Matching]
  A matching of a graph \(G\) is \textit{complete} if it contains all of \(G\)'s
  vertices. Sometimes, this is also called a \textit{perfect} matching.
\end{definition}

\begin{remark}
  If graph \(G\) has a matching \(M\) that contains all vertices in \(G\), we
  called \(M\) a perfect/complete matching. That is, a matching of size \(n/2\).
\end{remark}

If you are in Computer Science, your goal here would be to come up with an
algorithm to come up with the maximum size matching in \(G\) and then analyze
its running time. On the other hand, if you are in Mathematics, your goal would
be to find the condition for a perfect matching. Keep in mind that identifying
the perfect/maximum matching is an NP problem so it is very difficult to solve. 
