\begin{theorem}[Cayley's Theorem, 1889]
  There are \(n^{n-2}\) distinct labelled trees on \(n\) vertices.
\end{theorem}

The proof of this will be omitted due to Nawat not understanding
it. It might show up one day in the future.

% \begin{proof}[Proof with Pr\"{u}fer's Code]
%   We will show that there is a bijection \(f : X \to Y\) where \(X\) is the set
%   labelled tree of \(n\) vertices and \(Y\) is the \(n-2\) tuples of numbers 1 
%   to \(n\). is the set of Prufer's codes. Namely, we will show that
%   \(f(X) \subseteq Y\). Then, we will show that \(f^{-1}(Y) \subseteq X\).

%   [Left as exercise]
% \end{proof}

\section{Pr\"ufer Sequences}

Following are the SageMath code snippets for encoding and
decoding Pr\"ufer Sequences.

\begin{listing}[ht]
\begin{minted}{python}
def tree_to_prufer(T: Graph) -> List[int]:
    tmp = Graph()
    tmp.adj = T.adj.copy()
    T = tmp

    S = []
    leaves = {v + 1 for v in T.adj if T.degree(v + 1) == 1}
    while T.n() > 2:
        # gets the smallest leaf
        l = min(leaves)
        leaves.remove(l)

        # add parent to sequence
        [parent] = [v for v in T.N(l)]
        S.append(parent)
        T.remove_vertex(l)

        # update collection of leaves
        if T.degree(parent) == 1:
            leaves.add(parent)

    del T
    return S
\end{minted}
\caption{Encoding a tree into a Pr\"ufer Sequence}
\end{listing}

\begin{listing}[ht]
\begin{minted}{python}
def prufer_to_graph(S: List[int]) -> Graph:
    n = len(S) + 2
    L = [i + 1 for i in range(n)]

    T = Graph()
    for i, s in enumerate(S):
        # gets minimum vertex not in S and connect it to the graph
        v = min([l for l in L if l not in S[i:]])
        T.add_edge(v, s)
        L.remove(v)

    # adds the last edge/vertex
    assert len(L) == 2
    u, v = [l for l in L]
    T.add_edge(u, v)

    return T  
\end{minted}
\caption{Decoding a Pr\"ufer Sequence into a tree}
\end{listing}

\chapter{Matching and Covering}

In this chapter, the main question is as follows: Given a graph \(G\), how do we
find the matching \(M\) with maximum size?

\section{Core Definitions}

\begin{definition}[Edge Independent]
  A set \(F\) of edges in a graph is \textit{independent} if no two edges in
  \(F\) are adjacent.
\end{definition}

\begin{definition}[Matching]
  A \textit{matching} \(M\) of graph \(G\) is a subgraph of \(G\) such that
  every edge shares no vertex with any other edge. That is, each vertex in
  matching \(M\) has degree one.
\end{definition}

\begin{definition}[Complete/Perfect Matching]
  A matching of a graph \(G\) is \textit{complete} if it contains all of \(G\)'s
  vertices. Sometimes, this is also called a \textit{perfect} matching.
\end{definition}

\begin{remark}
  If graph \(G\) has a matching \(M\) that contains all vertices in \(G\), we
  called \(M\) a perfect/complete matching. That is, a matching of size \(n/2\).
\end{remark}

If you are in Computer Science, your goal here would be to come up with an
algorithm to come up with the maximum size matching in \(G\) and then analyze
its running time. On the other hand, if you are in Mathematics, your goal would
be to find the condition for a perfect matching. Keep in mind that identifying
the perfect/maximum matching is an NP problem so it is very difficult to solve. 
