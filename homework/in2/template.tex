\documentclass{article}

\input{preamble.tex}
\renewcommand{\Pr}[1]{\mathbf{Pr}\left[#1\right]}
\newcommand{\CPr}[2]{\mathbf{Pr}\left[#1\ |\ #2\right]}
\newcommand{\Ex}[1]{\mathbf{E}\left[#1\right]}
\newcommand{\ExSub}[2]{\mathbf{E}_#2\left[#1\right]}
\newcommand{\CEx}[2]{\mathbf{E}\left[#1\ |\ #2\right]}

\newcommand{\Exp}[1]{\text{Exp}\left(#1\right)}
\newcommand{\Pois}[1]{\text{Poisson}\left(#1\right)}
\newcommand{\DGamma}[2]{\text{Gamma}\left(#1, #2\right)}
\DeclareMathOperator{\diam}{diam}

\title{\Huge{{In-class Problem 2}}
	\\
\Large\scshape{{Graph Theory and Combinatorics}}}
\author{{Nawat Ngerncham}}
\date{\today}

\begin{document}

\maketitle

\begin{enumerate}
  \item What is the smallest number of edges that will ensure that a simple graph on 
    \(n\) vertices is Hamiltonian?
    \sol{
      The minimum number of edges that will ensure that a simple graph on
      \(n\) vertices is Hamiltonian is (by Hint) 
      \[ \binom{n-1}{2} + 2 \]

      First, we will show that it is possible for a graph with \(n\) vertices
      and \(\binom{n-1}{2} + 1\) edges to be non-Hamiltonian. First, let \(G =
      K_{n-1}\) graph. Next, \(G\) would have \(\binom{n-1}{2}\) edges and
      \(n-1\) graphs. Then, we add one more vertex, called \(v\), and connect 
      it to any vertex in the \(K_{n-1}\) portion of the graph. This makes the
      \(|E(G)| = \binom{n-1}{2}+1\) as desired. Notice that \(\deg v = 1\).
      Hence, we cannot form a cycle that visits \(v\) since if you
      \textit{enter} \(v\), then you would need to leave and visit the vertex
      it comes from again. Hence, it is possible that a graph with
      \(\binom{n-1}{2}+1\) vertex to be non-Hamiltonian.

      Now, it suffices to show that any graph with \(\binom{n-1}{2}+2\) edges is
      Hamiltonian.
      Claim that for every non-adjacent vertices \(u, v\), we have
      \begin{equation}\label{eqn:deg-cond}
        \deg u + \deg v \geq n
      \end{equation}
      If this claim is true, then the proposition at the beginning of the
      solution satisfies by Ore's Theorem.

      Let \(G\) be a simple graph of order \(n\) and \(\binom{n-1}{2}+2\) edges.
      Notice that the sum of degrees when there are \(\binom{n-1}{2}+2\) edges
      is
      \[
        \begin{aligned}
          2 \left[ \binom{n-1}{2}+2 \right] &= 2 \left[ \frac{(n-1)(n-2)}{2} + 2
          \right] \\
                                            &= (n-1)(n-2) + 4 \\
                                            &= n^2 - 3n + 6
        \end{aligned}
      \]
      by Euler's Handshaking Lemma.

      Assume to the contrary that there exists a pair of non-adjacent vertices
      \(u, v\) such that \(\deg v + \deg u < n\). Namely, that it violates 
      \ref{eqn:deg-cond}. Consider the 
      subgraph \(H\) of \(G\) where \(H\) is \(G\) without \(u, v\).
      It follows that \(H\) is a graph of order \(n-2\). We
      can then bound the sum of degrees in \(H\) as follows. Namely, this is
      computed by removing the edges incident to \(u\) or \(v\). 
      \[
        \begin{aligned}
          n^2 - 3n + 6 - \underbrace{2n}_\text{Lower bound}
            &< n^2 - 3n + 6 - \underbrace{2(\deg v + \deg u)}_\text{Actual
            removed} = \sum_{w \in V(H)} \deg w \\ 
          n^2 - 5n + 6 &< n^2 - 3n + 6 - 2(\deg v + \deg u)
        \end{aligned}
      \]
      Note that their neighborhoods are disjointed since they are non-adjacent.
      Notice also that the sum of degrees in a \(K_{n-2}\) graph would be 
      \[
        \begin{aligned}
          \sum_{w \in V(K_{n-2})} \deg w
            &= 2 \cdot \binom{n-2}{2} \\ 
            &= 2 \left(\frac{(n-2)(n-3)}{2}\right) \\ 
            &= (n-2)(n-3) \\ 
            &= n^2 - 5n + 6
        \end{aligned}
      \]
      Consider
      \[
        \begin{aligned}
          \underbrace{n^2 - 5n + 6}_{\min \sum_{w \in H} \deg w}
            &<& \underbrace{n^2 - 3n + 6 - 2(\deg v + \deg u)}_\text{Exact
            \(\sum_{w \in H} \deg w\)}
            &\leq& \underbrace{n^2 - 5n + 6}_{\sum_{w \in K_{n-2}} \deg w} \\ 
          -5n &<& -3n - (\deg v + \deg u) &\leq& - 5n
        \end{aligned}
      \]
      However, this is impossible because \(-5n \not{<} -5n\) for any \(n\) due
      to the strict inequality.
      This is a contradiction. Hence, it is impossible for any non-adjacent
      vertices \(u, v \in V(G)\) to be such that \(\deg v + \deg u < n\).
      Thus, every non-adjacent vertices \(u, v \in V(G)\) must be such that
      \(\deg u + \deg v \geq n\). Therefore, any graph with \(\binom{n-1}{2} +
      2\) edges must be Hamiltonian.
    }
\end{enumerate}

% in case of needing citations
% \bibliographystyle{plainurl}
% \bibliography{refs}

\end{document}
