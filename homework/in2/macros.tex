% from Stochastic Processes
\renewcommand{\Pr}[1]{\mathbf{Pr}\left[#1\right]}
\newcommand{\CPr}[2]{\mathbf{Pr}\left[#1\ |\ #2\right]}
\newcommand{\Ex}[1]{\mathbb{E}\left[#1\right]}
\newcommand{\ExSub}[2]{\mathbb{E}_#2\left[#1\right]}
\newcommand{\CEx}[2]{\mathbb{E}\left[#1\ |\ #2\right]}

\newcommand{\Exp}[1]{\text{Exp}\left(#1\right)}
\newcommand{\Pois}[1]{\text{Poisson}\left(#1\right)}
\newcommand{\DGamma}[2]{\text{Gamma}\left(#1, #2\right)}

% from Graph Theory
\DeclareMathOperator{\diam}{diam}


% === SELF-DEFINED ENVIRONMENTS OPERATORS ===
\newenvironment{nexample}{
  \begin{leftbar}
    \noindent\textbf{Example.}
}{
  \end{leftbar}
}

\newenvironment{distraction}{
    \begin{tcolorbox}[title=Distraction!,breakable,colframe=red!69!black,before upper={\parindent15pt}]
}{
    \end{tcolorbox}
}

\newenvironment{hwproblem}[2]{
	\noindent\textbf{Problem #1.} #2

	\begin{framed}
}{
	\end{framed}
}

\newmdenv[linewidth=0.5pt,linecolor=black,backgroundcolor=white]{singleframed}
\newenvironment*{singleframedindent}{
  \begin{singleframed}
    \setlength{\parindent}{\defparindent}\ignorespaces
  }{
  \end{singleframed}
}

\newcommand\sol[1]{
  \begin{singleframedindent}
    #1
  \end{singleframedindent}
}
