\begin{hwproblem}{6}{
    Let $G$ be a graph containing a cycle $C$, and assume that $G$
    contains a path of length at least $k$ between two vertices of $C$. Show 
    that $G$ contains a cycle of length at least $\sqrt{k}$. Is this best 
    possible?
  }
  Let \(P\) denote the path of length \(k\) between two vertices of \(C\). That
  is, suppose that \(C\) has length \(n\) and \(C = \{v_1, v_2, ..., v_n\}\),
  then there are \(1 \leq i, j \leq n\) such that \(i \neq j\) and \(P =
  \{v_i, p_1, p_2, ..., p_{k-2}, v_j\}\).

  Suppose that \(P\) intersects \(C\) at \(l\) different points including \(v_i,
  v_j\). There are two cases and we will show them one at a time.
  If \(l = 2\), then \(P\) only intersects \(C\) at its ends: \(v_i, v_j\).
  Then, we can form a cycle \(C'\) as follows:
  \[ C' = \{\underbrace{v_i, p_1, p_2, ..., p_{k-2}, v_{j}}_P, \underbrace{
  v_{j+1}, ..., v_{i-1}}_\text{Part of \(C\) between \(v_i, v_j\)}, v_i \} \]
  In this case, we will end up with a cycle \(C'\) of length at least \(k \geq
  \sqrt{k}\) since \(k \geq 1\).

  Otherwise, we have \(P\) intersects \(C\) at \(l > 2\) vertices. We will 
  collect this into a set denoted \(\Lambda\). Namely, let \(\Lambda
  = \{\lambda_1=v_i, \lambda_2, ..., \lambda_l=v_j\} = V(P) \cap V(C)\). If 
  \(l \geq \sqrt{k}\), then we are done and we can form the cycle from vertices 
  in \(\Lambda\) and the points between vertices between them, either from \(C\)
  or \(P\). That is, the cycle \(C'\) is defined as follows.
  \[
    C' = \{ \lambda_1 = v_i, \underbrace{v_{i+1}, ...}_\text{Part of \(C\)}, 
          \lambda_2, \underbrace{p_{\alpha}, ...}_\text{Part of \(P\)}, \lambda_3,
          ..., \lambda_l = v_j, ..., v_i=\lambda_1 \}
  \]
  Otherwise, \(l < \sqrt{k}\). Claim that in a path of length \(k\), if we
  mark the ends of the path and some other vertices totalling \(l\) vertices 
  in the path, then there exists a segment of the path between two adjacent 
  markers (including the markers themselves) that is at least of length \(k/l\).
  We will prove this later. Let the markers represent the vertices where \(P\)
  intersects \(C\). Then, we can come up with the following inequality.
  \[ l < \sqrt{k} \iff \frac{k}{l} > \frac{k}{\sqrt{k}} = \sqrt{k} \]
  Then, we can form a cycle \(C'\) with length at least \(\sqrt k\) as follows.
  Suppose that the segment that is at least \(\sqrt k\) in length are between
  intersecting vertices \(\lambda_{i'}\) and \(\lambda_{j'}\).
  \[ C' = \{\lambda_{i'}, \underbrace{p_{\alpha}, p_{\alpha+1}}_\text{Part of
  \(P\)}, ..., \lambda_{j'}, \underbrace{v_{\beta}, v_{\beta+1}, ...}_\text{Part
  of \(C\)}, \lambda_{i'}\} \]
  Hence, we end up with a cycle \(C'\) of at least length \(\sqrt k\).
  Therefore, we can conclude that if a graph \(G\) containing a cycle \(C\) and a
  path \(P\) of length at least \(k\) between two vertices of \(C\), then \(G\)
  contains a cycle of at least length \(\sqrt k\).

  \begin{proof}[Proof of \(k/l\) Claim]
    Consider the path \(P = \{u_1, u_2, ..., u_k\}\). Then, we denote the
    \(i\)-th marked vertex in \(P\) as \(\lambda_i\). Define a \textit{segment
    starting at marked vertex \(\lambda_i\)}, denoted \(S(\lambda_i)\),
    to be the path between \(\lambda_{i}\) and \(\lambda_{i+1}\) for some
    \(i\). Namely, a segment starting at the marked vertex \(\lambda_i\) would
    be \(\{\lambda_i, u_{\alpha}, u_{\alpha+1}, ..., \lambda_{i+1}\}\).
    Notice that if we sum the number of vertices of every segment, we would have
    the following.
    \[ \sum_{i=1}^{l-1} S(\lambda_i) = |V(P)| + (l - 2) = k+l-2 \]
    We get \(|V(P)|\) from the length of \(P\) and we get \((l-2)\) since we 
    count every marked vertex twice except for the start and end.

    Assume to the contrary that every segment between any pair of
    adjacent markers have length \(< k/l\). Then, the sum of the number of
    vertices in every segment would be
    \[
      \begin{aligned}
        \sum_{i=1}^{l-1} S(\lambda_i) &< (l-1)\frac{k}{l} \\
                                      &< k - \frac{1}{l} \neq k+l-2
      \end{aligned}
    \]
    Therefore, there must be at least one segment with length at least \(k/l\).
  \end{proof}

  As for improving the bound, we can always use the reasoning above to obtain
  the minimum bound of \(k/l\) where \(l\) is the number of intersections using
  the reasoning above. However, this bound only works if we are willing count
  the number of intersections.

  If we want to only have \(k\) in our bound, then I don't know :/ Aj. and Bhum said
  \(\sqrt{2k}\) and believe them though.
\end{hwproblem}
