\begin{hwproblem}{4}{
    Let $G$ be a connected $r$-regular graph of even order $n$ such that 
    $\bar{G}$ is also connected. Show that
    \begin{enumerate}[label=(\alph*)]
      \item either $G$ or $\bar{G}$ is Eulerian.
      \item either $G$ or $\bar{G}$ is Hamiltonian.
    \end{enumerate}
  }
  We will assume that \(G\) is a simple graph to prove both parts.
  Otherwise, it would not have a complement graph \(\overline{G}\).

  First, we will show that either \(G\) or \(\overline{G}\) is Eulerian. Namely,
  we will show that every vertex in either \(G\) or \(\overline{G}\) must have
  an even degree. Recall that for integers \(r, n\) such that \(0 \leq r \leq
  n-1\), if there is an \(r\)-regular graph of order \(n\), then at least one of
  \(r\) or \(n\) is even. Notice that \(0 \leq r \leq n-1\) is already satisfied
  since a vertex in any simple graph cannot have more than \(n-1\) edges.
  If \(r\) is even, then we are done. Hence, \(G\) is connected (by
  assumption) and every vertex in \(G\) has an even degree. Thus, we have that
  \(G\) is Eulerian.

  Otherwise, \(r\) is odd and \(n\) is even. Observe that 
  \(\overline{G}\) is an \((n-r-1)\)-regular graph since any vertex can have a
  degree of up to \(n-1\) and \(r\) is already in \(G\), leaving \(n-r-1\) left
  for \(\overline{G}\). Notice also that \(n-r-1\) is even since its even minus
  odd minus odd. Hence, \(\overline{G}\) is connected (by assumption) and 
  every vertex in \(\overline{G}\) has an even degree. Thus, we have that
  \(\overline{G}\) is Eulerian. Therefore, we can conclude that if a connected
  \(r\)-regular graph of even order \(n\) such that \(\overline{G}\) is also
  connected, then either \(G\) or \(\overline{G}\) is Eulerian.

  Now, we will show that either \(G\) or \(\overline{G}\) is Hamiltonian.
  Recall that if \(G\) is an \(r\)-regular graph, then \(\overline{G}\) is an
  \((n-r-1)\)-regular graph. If \(r \geq n/2\), then we are done and \(G\) is
  a simple connected graph with \(\deg v \geq n/2\) for all \(v \in V(G)\). By
  Dirac's Theorem, \(G\) is Hamiltonian.

  Otherwise, consider the following.
  Since \(n\) is even, \(n/2\) is an integer. Hence, we claim that 
  \[ r < \frac{n}{2} \iff r \leq \frac{n}{2} - 1 \]
  To show this, we can denote \(n = 2n_0\) since \(n\) even. It follows that 
  \[ r < \frac{n}{2} = n_0 \iff r \leq \frac{n}{2} - 1 = n_0 - 1 \]
  since \(r, n, n_0 \in \mathbb{Z}\). Consequently, we have the following
  \[
    \begin{aligned}
      (n-r-1) + (n-r-1) &= 2n - 2(r-1) \\ 
                        &= n + \left[n - 2(r-1)\right] \\ 
                        &\geq n + (n - n) = n \\
    \end{aligned}
  \]
  Hence, we have that \(\overline{G}\) is a simple connected graph with 
  \((n-r-1) + (n-r-1) = \deg v + \deg u \geq n\) for every \(u, v \in 
  V(\overline{G})\). By Ore's Theorem, \(G\) is Hamiltonian. Therefore, we
  can conclude that if a connected \(r\)-regular graph of even order \(n\) such
  that \(\overline{G}\) is also connected, then either \(G\) or \(\overline{G}\)
  is Hamiltonian.
\end{hwproblem}
