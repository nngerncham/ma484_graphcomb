\begin{hwproblem}{1}{
    Prove that if the numbers $1,2, \ldots, 12$ are randomly positioned around a 
    circle, then some set of three consecutively positioned numbers must have a 
    sum of at least 20.
  }
  [Source \cite{p1}]
  Let us first compute the sum of every set of three consecutively positioned
  numbers. Notice that each number shows up in 3 sets each. Hence, the sum can
  be computed by
  \[ \left(\sum_{i=1}^{12} i\right) \cdot 3 = 78 \cdot 3 = 234 \]

  Let us now show that it is impossible to have no set of three consecutively
  positioned numbers sum up to at least 20. That is, the sum of every three
  consecutively positioned numbers must add up to at most 19. Hence, the upper
  bound for the sum of every set of three consecutive numbers must is 
  \[ 19 \cdot 12 = 228 \]
  since there are 12 sets of three consecutively positioned numbers. Namely, if
  we denote a set of three consecutive numbers by its center, then we would have
  12 centers since we have 12 numbers. However, since the sum of every set of
  three consecutive numbers must sum up to 234, there must at least one set of
  three consecutive numbers whose sum is at least 20 to make up for the 
  difference.
\end{hwproblem}
