\setcounter{question}{5}
\question If \(G\) is connected and \(\Delta\)-regular with a
  cut-vertex \(x\), then \(\chi^{\prime}(G)>\Delta\).
\begin{solution}
  Let \(n\) be the number of verties in \(G\). We will show this
  by case. For this problem, we will use \(\Delta\) and
  \(\Delta(G)\) interchangeably. This does not affect the result
  since \(G\) is a \(\Delta\)-regular graph so \(\Delta =
  \Delta(G)\) anyway.

  \begin{enumerate}[label=\Roman*.]
    \item If \(n\) is odd, let \(m = |E(G)|\). Then, we have 
      \[
        \begin{aligned}
          2m &= \sum_{v \in V(G)} \deg v \\
          2m &= n \cdot \Delta \\
          m  &= \frac{n \cdot \Delta}{2} \\
             &> \frac{(n-1) \cdot \Delta}{2} 
        \end{aligned}
      \]
      Therefore, by Theorem 14, we have that \(\chi'(G) =
      \Delta + 1 > \Delta\).

    \item Otherwise, \(n\) is even. Suppose that \(G - x\) is
      separated into components \(G_1, G_2, \ldots, G_k\).
      Note that
      \[ \sum_{i=1}^k |V(G_i)| = n-1 \]
      So, there must be a component, say \(G_i\), that has an odd
      number of vertices in it. Let \(l\) be the number of edges
      in \(G_i\)
      that is incident to \(x\) and observe that the following 
      satisfies: \(0 < l < \Delta\) since if \(l = 0\), then \(G\)
      is disconnected; and if \(l = \Delta\), then \(x\) is not
      connected to any other component.
      Observe also that since \(x\) is a cut vertex, every edge in
      \(G_i\) that is not incident to \(x\) is incident to
      another vertex in \(G_i\).
      Let \(H\) be the subgraph of \(G\) induced
      by \(G_i\) and let \(s = |V(H)|\). Consider
      \[
        \begin{aligned}
          |E(H)| &= \frac{\sum_{v \in H} \deg_H v}{2} \\
                 &= \frac{s \cdot \Delta - l}{2} \\
                 &> \frac{(s-1)\cdot\Delta}{2}
        \end{aligned}
      \]
      Note that \(l\) is not double-counted here since \(x\) is
      not part of the summation.
      By Theorem 14, we have that the \(\chi'(H) = \Delta+1\).
      Therefore, since \(H \subseteq G\), \(\chi'(G) = \Delta+1 >
      \Delta\) as well.
  \end{enumerate}
\end{solution}
