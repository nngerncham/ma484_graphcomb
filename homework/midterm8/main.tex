% === HOMEWORK PORTION ===
\documentclass[answers]{exam}

% BASE TAKEN FROM ICCS315 SCRIBE NOTES

% --- SETUP STUFF ---
\usepackage[a4paper, margin=1in]{geometry}
\usepackage{enumitem}
\usepackage{booktabs}

\usepackage{url}
\usepackage[unicode]{hyperref}

\setcounter{secnumdepth}{2}

% --- MATH STUFF ---
\usepackage{amsthm, amsmath, amssymb}
\usepackage{mathtools,xspace}
\usepackage{nicefrac}

\usepackage{bbm}
\usepackage{dsfont}
\usepackage{cancel}

\usepackage{blkarray}
\newcommand{\matindex}[1]{\mbox{\scriptsize#1}} % Matrix index

% --- FONT STUFF ---
% Has to be under math stuff for some reason :/
\usepackage{newpxtext, newpxmath}
\usepackage[T1]{fontenc}

% --- DIAGRAM STUFF ---
\usepackage{tikz,pgfplots,xcolor,graphicx}
\usepackage{graphicx}
\usepackage{colortbl}
\usepackage{caption}
\usepackage{subcaption}

\usepackage[breakable,skins]{tcolorbox}
\usepackage{framed}
\usepackage{float}

\pgfplotsset{compat=1.18}

% --- THEOREM STUFF ---
\newtheorem{theorem}{Theorem}[section]
\newtheorem{proposition}[theorem]{Proposition}
\newtheorem{lemma}[theorem]{Lemma}
\newtheorem{corollary}[theorem]{Corollary}

\theoremstyle{definition}
\newtheorem{definition}[theorem]{Definition}
\newtheorem{example}[theorem]{Example}

\theoremstyle{remark}
\newtheorem{remark}[theorem]{Remark}
\newtheorem{claim}[theorem]{Claim}
\newtheorem{fact}[theorem]{Fact}

\usepackage{algorithm}
\usepackage[indLines=true]{algpseudocodex}
\usepackage{algorithmicx}
\algnewcommand\algorithmicinput{\textbf{Input:}}
\algnewcommand\Input{\item[\algorithmicinput]}
\algrenewcommand\algorithmicoutput{\textbf{Output:}}
\algrenewcommand\Output{\item[\algorithmicoutput]}
\algrenewcommand\algorithmicrequire{\textbf{Require:}}
\algrenewcommand\Require{\item[\algorithmicrequire]}

\renewcommand{\Pr}[1]{\mathbf{Pr}\left[#1\right]}
\newcommand{\CPr}[2]{\mathbf{Pr}\left[#1\ |\ #2\right]}
\newcommand{\Ex}[1]{\mathbf{E}\left[#1\right]}
\newcommand{\ExSub}[2]{\mathbf{E}_#2\left[#1\right]}
\newcommand{\CEx}[2]{\mathbf{E}\left[#1\ |\ #2\right]}

\newcommand{\Exp}[1]{\text{Exp}\left(#1\right)}
\newcommand{\Pois}[1]{\text{Poisson}\left(#1\right)}
\newcommand{\DGamma}[2]{\text{Gamma}\left(#1, #2\right)}
\DeclareMathOperator{\diam}{diam}

\renewcommand{\questionlabel}{\textbf{Problem \thequestion.}}
\renewcommand{\solutiontitle}{}

\title{\Huge{Midterm Problem 8 Redo}
	\\
\Large\scshape{Graph Theory and Combinatorics}}
\author{Nawat Ngerncham (6380496)}
\date{\today}

\begin{document}

\maketitle

\begin{questions}
  \setcounter{question}{7}
  \question Prove that a graph \(G\) is 2-connected if and only if
  for every ordered triple \((x, y, z)\) of vertices, there is an
  \(x-z\) path through \(y\).
  \begin{solution}
    First, we will show the forward direction: If a graph is
    2-connected, then for every ordered triple \((x, y, z)\) of
    vertices, there is an \(x-z\) path through \(y\). Fix \(x,
    z\) to be distinct vertices in \(G\). Without loss of
    generality, there must be \(x-y\) and \(x-z\) paths since the
    graph is connected. Now, consider \(y\). If \(y, z\) are
    adjacent, then we are done and we have a \(x-z\) path that
    goes through \(y\).

    Otherwise, there are at least 2 internally disjoint \(y-z\)
    paths since the graph is 2-connected by Menger's Theorem.
    Then, there is a \(y-z\) path that does not go through \(x\)
    by the Pigeonhole Principle. Let us call this path \(P\).
    Hence, there is a \(x-z\) path that goes through \(y\) by
    concatenating the original \(x-y\) path and \(P\). Therefore,
    if a graph \(G\) is 2-connected, then for every ordered
    triple \(x, y, z\) in \(G\), there is an \(x-z\) path through
    \(y\).

    Next, we will show the converse direction: If for every
    ordered triple \(x, y, z\) of vertices, there is an \(x-z\)
    path through \(y\) in a graph \(G\), then \(G\) is
    2-connected. We will show this by contrapositivity:
    Assume that \(G\) is not 2-connected (i.e., \(G\) is 
    1-connected). Then, we will show there are \(x, y, z \in 
    V(G)\) such that that there is no \(x-z\) path
    through \(y\) in a graph \(G\).

    Since \(G\) is 1-connected, there is a cut vertex. Fix \(z\)
    to be a cut vertex and let \(x\) and \(y\) belong to 
    different components in \(G-z\). Note that \(G-z\) must have
    two components. Otherwise, \(G-z\) would still be connected
    and \(z\) would not be a cut vertex. Consequently, there
    would be no \(x-z\) path through \(y\).
  \end{solution}
\end{questions}

% % in case of needing citations
% \bibliographystyle{plainurl}
% \bibliography{refs}

\end{document}
