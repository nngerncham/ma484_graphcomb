\documentclass[answers]{exam}

\input{preamble.tex}
\renewcommand{\Pr}[1]{\mathbf{Pr}\left[#1\right]}
\newcommand{\CPr}[2]{\mathbf{Pr}\left[#1\ |\ #2\right]}
\newcommand{\Ex}[1]{\mathbf{E}\left[#1\right]}
\newcommand{\ExSub}[2]{\mathbf{E}_#2\left[#1\right]}
\newcommand{\CEx}[2]{\mathbf{E}\left[#1\ |\ #2\right]}

\newcommand{\Exp}[1]{\text{Exp}\left(#1\right)}
\newcommand{\Pois}[1]{\text{Poisson}\left(#1\right)}
\newcommand{\DGamma}[2]{\text{Gamma}\left(#1, #2\right)}
\DeclareMathOperator{\diam}{diam}

% === NOTES PORTION ===
% \begin{document}
%
% \frontmatter
% \thispagestyle{empty}

\begin{center}
\vspace{5cm}

\Huge{Graph Theory}

\vspace{5em}
\begin{center}
  \includegraphics[height=0.5\textheight]{figures/chad.png}
\end{center}
\vspace{5em}

\Large Student's Notes for {T2/2023-2024}

\Large Written by {Nawat Ngerncham}

\Large Last updated: \today
\end{center}

%
% \setcounter{tocdepth}{1}
% \tableofcontents
%
% \mainmatter

% Something something

% in case of needing citations
% \bibliographystyle{plainurl}
% \bibliography{refs}

% \end{document}

% === HOMEWORK PORTION ===
\renewcommand{\questionlabel}{\textbf{Problem \thequestion.}}
\renewcommand{\solutiontitle}{}

\title{\Huge{{In-class Problem 3}}
	\\
\Large\scshape{{Graph Theory and Combinatorics}}}
\author{{Nawat Ngerncham}}
\date{\today}

\begin{document}

\maketitle

\begin{questions}
  \question 
    Take a regular deck of 52 playing cards and randomly deal them into 13 
    piles of 4 cards each. Show that, there exists a way to pull out one card
    from each pile so that you have a card from every rank (ace, 2, ..., king).

    \begin{solution}
      First, let us model this problem into a graph problem. Let us construct a
      bipartite graph \(G\) whose vertices are split into sets \(A\) and \(B\)
      as follows.
      \begin{enumerate}
        \item Let each vertex in \(A\) represent a rank of cards. Namely, we
          have \(A = \{a, 2, 3, ..., 10, j, q, k\}\) where \(a\) is the Ace 
          and \(j, q, k\) are ranks Jack, Queen, and King, respectively.
        \item Let each vertex of \(B\) represent each pile of cards. Namely, 
          we have \(B = \{P_i : i = 1, 2, ..., 13\}\). 
        \item Add edges such that for each vertex \(v\) that represents rank 
          \(r\), \(v\) is adjacent to \(P_i\)'s in \(B\) if at least one card of
          rank \(r\) is present in \(P_i\). 
      \end{enumerate}

      We claim that for any \(S \subseteq A\), we have that
      \[ |N(S)| \geq |S| \]
      Then, by Hall's Marriage Theorem, there is a matching of \(A\) in \(G\).
      This means that each rank (vertex in \(A\)) will be matched to a pile
      (vertex in \(B\)) that it can be drawn from. Thus, we would have that
      there always exists a way to pull out one card from each pile such that we
      have a card from every rank. We are now only left with proving our claim.

      Assume to the contrary that there exists \(S \subseteq A\) such that the
      \(|S| > |N(S)|\). Let \(s = |S|\) and \(k = |N(S)|\). Since each pile of
      cards has 4 cards, the total number of cards that show up in any
      pile of \(N(S)\) would be \(4k\). That is, 
      \[ \sum_{P_j \in N(S)} |P_j| = 4k \]

      Recall that in a regular deck of cards, each rank has 4 cards, one from
      each suit (Clubs, Spades, Hearts, Diamonds). So, the number of cards that
      has a rank that shows up in \(S\) is \(4s\). Since \(k < s\), it follows
      that \(4k < 4s\). That is, the total number of cards that show up in the
      piles in \(N(S)\) would be less than the number of cards that have the 
      ranks in \(S\). Since \(N(S)\) is made up of piles of cards that have any 
      rank in \(S\), this would mean that some cards are lost. This is a
      contradiction since we are splitting cards without throwing any away. 

      Therefore, we can conclude that if we were to 
      take a regular deck of 52 playing cards and randomly deal them into 13 
      piles of 4 cards each, then there exists a way to pull out one card
      from each pile so that we have a card from every rank.
    \end{solution}
\end{questions}

% in case of needing citations
% \bibliographystyle{plainurl}
% \bibliography{refs}

\end{document}
