\question Let \(G\) be connected graph of order \(n\) where
\(\Delta(G)+\delta(G) \geq n-2\). Show that \(\operatorname{diam}
(G) \leq 4\).

\begin{solution}
  We will show that for every pair of distinct vertices \(u, v\),
  \(d(u, v) \leq 4\) in cases. Let \(x\) be the vertex of maximum degree.
  Without loss of generality, we assume that \(u \neq x \neq v\). Otherwise, 
  any path that goes through \(x\) will be cut short, making the distance
  shorter. 

  \begin{enumerate}
    \item \(u, v\) are adjacent. Then, we would have that \(d(u, v) = 1\).

    \item Both \(u, v \in N(x)\) or \(N(u) \cap N(v) \neq \emptyset\).
      Then, we can make path \(u-x-v\) or \(u-w-v\) for some \(w \in
      N(u) \cap N(v)\), respectively. This results in \(d(u, v) = 2\).

    \item \(N(u) \cap N(x) = \emptyset\). We will have two subcases for this.
      \begin{enumerate}
        \item If \(u \notin N(x)\), then we claim that either \(v
          \in N(u)\) or \(v \in N(x)\). Since \(\Delta + \delta \geq
          n-2\), we also have that \(\deg u + \deg x \geq n-2\). Furthermore, since
          \(u, v\)'s neighborhoods are disjoint, it follows that
          \[
          \begin{aligned}
            \left|\{u\} \cup N(u) \cup \{x\} \cup N(x)\right|
              &= 1 + \deg u + 1 + \deg x \\
              &\geq 1+1+n-2 \\
              &= n
          \end{aligned}
          \]
          We can see that \(u\), \(u\)'s neighbors, \(x\), and \(x\)'s neighbors
          make up every vertex in the graph. Thus, \(v\) must
          belong to either neighborhood. Since we already considered the case that
          \(v \in N(u)\), we will show the case that \(v \in N(x)\).

          We claim that there
          must be vertices \(w \in N(u)\) and \(y \in N(x)\) such that \(w, y\) are
          adjacent. If there aren't, then the graph would
          be disconnected, contradicting our assumption. Thus, we can
          create a path \(u - w - y - x - v\). Hence, we would have that \(d(u, v) =
          4\).
        \item Otherwise, \(u \in N(x)\). Then, we have
          \[
            \begin{aligned}
              \left|\{u\} \cup N(u) \cup \{x\} \cup N(x)\right|
                &= 1 + \deg u + \deg x \\
                &\geq 1+n-2 \\
                &= n-1
            \end{aligned}
          \]
          Let \(s = \left|\{u\} \cup N(u) \cup \{x\} \cup N(x)\right|\). If \(s =
          n\), this forces \(v\) to be in either neighborhood.
          Then, we either have \(d(u, v) = 1\) when \(v \in N(u)\) or \(d(u, v)
          = 2\) when \(v \in N(x)\) through the path \(u-x-v\).
          Similarly, if \(s=n-1\) with \(v \in N(u)\) or \(N(x)\), then we would
          have the same results.

          Otherwise, \(s=n-1\) and \(v \notin N(u)\) nor \(N(x)\). So, \(v\)
          would be the only vertex not in \(\{u\} \cup N(u) \cup \{x\} \cup N(x)\).
          Then, there must be \(w \in N(v)\) such that \(w \in N(u)\) or
          \(N(x)\) as well. Otherwise, the graph would be
          disconnected. So, we would have that \(d(u, v) = 2\) when
          \(w \in N(u)\) through path \(u-w-v\) and
          \(d(u, v) = 3\) when \(w \in N(x)\) through path \(u-x-w-v\).
      \end{enumerate}

    \item For this final case, we are left with when
      \begin{itemize}
        \item \(u, v\) are not adjacent (opposite of case 1),
        \item both \(u, v \notin N(x)\) and \(N(u) \cap N(v) = \emptyset\)
          (opposite of case 2), and
        \item \(N(u) \cap N(x) \neq \emptyset\) (opposite of case 3).
      \end{itemize}
      We will have two subcases for this (I know it's already a lot but please
      bear with me).

      \begin{enumerate}
        \item If \(u \notin N(x)\), we claim that \(N(v) \cap
          N(x) \neq \emptyset\) as well. Consequently, we can construct the
          path \(u - w - x - y - v\) where \(w \in N(u) \cap N(x)\) and \(y \in N(x)
          \cap N(v)\). It then follows that \(d(u, v) = 4\).
          Assume to the contrary that \(N(v) \cap N(x) = \emptyset\).
          Then, we would have that \(\deg v + \deg x \geq \delta + \Delta \geq
          n-2\). Notice that
          \[
            \begin{aligned}
              \left|\{v\} \cup N(v) \cup \{x\} \cup N(x)\right|
                &= 1 + \deg u + 1 + \deg x \\
                &\geq 1 + 1 + n-2 \\
                &= n
            \end{aligned}
          \]
          Again, we can see that \(v\), its neighbors, \(x\), and its neighbors
          make up every vertex of the graph. Then, we would have that \(u\) must
          belong to either neighborhood as previously shown. This is a
          contradiction since we assume that \(u \notin N(v)\) and \(u \notin
          N(x)\). Thus, \(N(v) \cap N(x) \neq \emptyset\).

        \item Otherwise, \(u \in N(x)\), making \(d(u, x) = 1\).
          Then, we just need to show that \(d(v,
          x) \leq 3\) and we would end up with \(d(u, v) \leq 4\).

          If \(N(v)\) and \(N(x)\) are disjoint, then, as shown multiple times
          previously, \(\{v\} \cup N(v) \cup \{x\} \cup N(x)\) would cover every
          vertex of the graph. Consequently, there must be \(w \in N(v)\) and
          \(y \in N(x)\) that are adjacent to each other. So, there must be path
          of length 3: \(v-w-y-x\). Hence, \(d(x, v) = 3\) and \(d(u, v) = 1 +
          d(x, v) = 4\).

          Otherwise, \(N(v) \cap N(x) \neq \emptyset\). Then, there must be \(w
          \in N(v) \cap N(x)\) that allows us to construct the path \(x-w-v\).
          Hence, \(d(x, v) = 2\) and \(d(u, v) = 1 + d(x, v) = 3\).
      \end{enumerate}
  \end{enumerate}

  Therefore, we can conclude that for any connected graph \(G\)
  of order \(n\) where \(\Delta(G) + \delta(G) \geq n-2\),
  \(\diam(G) \leq 4\) satisfies. 

  I'm sorry if this solution has a lot of repeated cases/arguments. I got really
  paranoid while writing this up so I felt like I needed to cover everything
  even though a lot of them are repeated.
\end{solution}
