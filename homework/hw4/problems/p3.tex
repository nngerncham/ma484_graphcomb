\question Derive the Hall's marriage theorem from König's theorem.

\begin{solution}
  % \begin{lemma}
  %   Let \(G\) be a bipartite graph whose vertices are separated in to sets \(A\)
  %   and \(B\) such that \(|A| \leq |B|\). Any matching \(M\) that contains every
  %   vertex in \(A\) is maximal.
  % \end{lemma}
  %
  % \begin{proof}
  %   Suppose to contradict that
  %   \(M\) contains every vertex in \(A\) but is not maximal. Then, there must be
  %   more vertices in \(B\) that are adjacent to each other for them to match
  %   with each other. However, no vertices in \(B\) can be adjacent to each
  %   other since \(G\) is bipartite. Therefore, \(M\) must be maximal.
  % \end{proof}
  
  \begin{lemma}
    If \(C\) is a minimal vertex cover, then the marriage condition satisfies.
  \end{lemma}

  \begin{proof}
    Suppose to the contrary that there exists \(S \subseteq C\) such that \(|S|
    > |N(S)|\). Then, there must be at least one vertex \(x \in S\) whose
    neighborhood is also \(N(S-x)\). That is, there is \(x\) such that \(N(S)
    = N(S-x)\). Hence, we can remove that vertex the vertices in \(N(S-x)\)
    would make up edges incident to \(x\) (and more) anyway. This implies that
    \(C-x\) would still be a minimal vertex cover. This is a contradiction since
    we already assume that \(C\) is a minimal vertex cover already. Therefore,
    we can conclude that if \(C\) is a minimal vertex cover, then the marriage
    condition satisfies.
  \end{proof}

  We will now prove Hall's Marriage Theorem using K\"onig's Theorem. First, for
  the forward direction: If \(G\) has a matching of \(A\), then \(|S| \leq
  |N(S)|\) for every \(S \subseteq A\). Note that the matching of \(A\) is
  maximal. If not, then there must be more vertices in \(B\) that are adjacent
  to each other---an impossibility. So, the the cardinality of the minimum
  vertex cover is \(|A|\) by K\"onig's Theorem. Since \(G\) is bipartite, every
  vertex of \(A\) must be incident to every edge of \(G\). Hence, \(A\) is also
  a minimum vertex cover of \(G\) as well since it also has size \(|A|\). Since
  \(A\) is a minimal vertex cover, then marriage condition satisfies as by the
  lemma above.

  Now, we will show the reverse direction: If \(|S| \leq |N(S)|\) for every \(S
  \subseteq A\), then \(G\) has a matching of \(A\). Suppose to contradict that
  the marriage condition satisfies but \(G\) has no matching of \(A\). Then, the
  maximal matching must have size less than \(|A|\). By K\"onig's Theorem, the
  minimal vertex cover has size strictly less than \(|A|\) as well. Let \(C = A'
  \cup B'\) be the minimal vertex cover for some \(A' \subseteq A\) and \(B'
  \subseteq B\). It follows that \(|C| = |A'| + |B'|\) and
  \[
    \begin{aligned}
      |A'| + |B'| &< |A| \\
      |B'| &< |A| - |A'| \\
           &= |A \setminus A'|
    \end{aligned}
  \]
  Since \(A'\) is part of the vertex cover, \(A \setminus A'\) are only adjacent
  to elements in \(B'\) since vertices in \(B'\) are incident with the remaining
  edges of \(G\). Consequently, we have that
  \[ |N(A \setminus A')| \leq |B'| < |A \setminus A'| \]
  This is a contradiction since we assumed that the marriage condition satisfies.
  Hence, we have that if the marriage condition satisfies, then there exists a
  matching of \(A\). 
\end{solution}
