% \section{Cycles}

% \begin{frame}[fragile]{Approximating Number of Triangles}
    
%     \begin{verbatim}
%         def sample_one(G):
%             e = choose edge u.a.r from E
%             v = choose vertex u.a.r from V
%             if v connects to endpoints of e:
%                 return 1
%             return 0
            
%         def count_triangles(G, T):
%             num_triangles = 0
%             for i = 1 up to T:
%                 num_triangles += sample_one(G)
%             return num_triangles
%     \end{verbatim}

% \end{frame}

% \begin{frame}{Sampling Size}
%     \begin{theorem}[Sampling Size]
%         \label{triangle-sampling-size}
%         Let $\epsilon > 0$, $0 < \delta \leq 1/2$, $\tau$ be the actual number of triangles in $G$ on $n$ vertices and $m$ edges, and $T \geq \frac{2\tau + \epsilon}{\epsilon} \ln \left(\frac{2}{\delta}\right)$. Then,
%         \[\textbf{Pr}[|\hat{\tau} - \tau| < \epsilon] \geq 1 - \delta\]
%         In human language, the probability that the estimate will be at most $\epsilon$ off is at least $1 - \delta$. 
%     \end{theorem}
%     \vspace{0.2in}
%     To prove this theorem, we will need to:
%     \begin{enumerate}
%         \item prove two other lemmas
%         \item use Chernoff's bounds
%     \end{enumerate}
% \end{frame}

% \begin{frame}{Analysis}
%     \begin{lemma}[Expected Number of Triangles]
%         \label{expected-triangle}
%         Let $\alpha_i = \alpha_1, \alpha_2, \cdots, \alpha_T, \hat{\alpha} = 1/T \sum_{i = 1}^T \alpha_i$, and $\hat{\tau} = \frac{m(n - 2)}{3}\hat{\alpha}$ be the estimated number of triangles. Then,
%         \[\mathbb{E}[\hat{\tau}] = \tau\]
%     \end{lemma}
%     \begin{lemma}[Expected Triangle Estimator]
%         \label{expected-triangle-estimator}
%         Let $\alpha$ be the indicator random variable representing the results of sample\_one subroutine and $G$ be a graph on $n$ vertices and $m$ edges. Then,
%         \[\mathbb{E}[\alpha] = \frac{3\tau}{m(n - 2)}\]
%     \end{lemma}
% \end{frame}

% \begin{frame}{Expected Triangle Estimator}
%     \begin{proof}[Expected Triangle Estimator]
%         Let $u, v, w$ be the vertices of any triangle. Let $A$ be the event where the edge picked u.a.r has $u$ and $v$ as the end vertices. Also, let $B$ be the event where $w$ is a neighbor of $u$ and $v$. Then,
%         \[\textbf{Pr}[A] = \frac{3\tau}{n}, \textbf{Pr}[B] = \frac{1}{n - 2}, \textbf{Pr}[A \cap B] = \frac{3\tau}{m(n - 2)}\]
%         Since $\mathbb{E}[\alpha] = \textbf{Pr}[\alpha = 1] = \textbf{Pr}[A \cap B]$, then,
%         \[\mathbb{E}[\alpha] = \textbf{Pr}[A \cap B] = \frac{3\tau}{m(n - 2)}\]
%     \end{proof}
% \end{frame}

% \begin{frame}{Expected Number of Triangles}
%     \setlength{\abovedisplayskip}{0pt}
%     \setlength{\belowdisplayskip}{0pt}
%     \setlength{\abovedisplayshortskip}{2pt}
%     \setlength{\belowdisplayshortskip}{2pt}
%     \begin{proof}[Expected Number of Triangles]
%         \begin{align*}
%             \intertext{Using the definition of expectation, we obtain,}
%             \mathbb{E}[\hat{\tau}] &= \mathbb{E}\left[\frac{m(n - 2)}{3}\hat{\alpha}\right] = \frac{m(n - 2)}{3}\mathbb{E}[\hat{\alpha}]\\
%             &= \frac{m(n - 2)}{3} \cdot \frac{1}{T}\sum_{i = 1}^T\mathbb[\alpha_i]\\
%             \intertext{Using lemma \ref{expected-triangle-estimator} we get,}
%             &= \frac{m(n - 2)}{3} \cdot \frac{1}{T} \cdot T \cdot \frac{3\tau}{m(n - 2)}\\
%             &= \tau
%         \end{align*}
%     \end{proof}
% \end{frame}

% \begin{frame}{Proof of Theorem \ref{triangle-sampling-size}}
%     \setlength{\abovedisplayskip}{0pt}
%     \setlength{\belowdisplayskip}{-4pt}
%     \setlength{\abovedisplayshortskip}{0pt}
%     \setlength{\belowdisplayshortskip}{0pt}
%     \begin{proofs}[Minimum Sampling Size]
%         \begin{align*}
%             \textbf{Pr}[|\hat{\tau} - \tau| < \epsilon] \geq 1 - \delta &\Longleftrightarrow 1 - \textbf{Pr}[|\hat{\tau} - \tau| \geq \epsilon] \geq 1 - \delta\\
%             &\Longleftrightarrow \delta \geq \textbf{Pr}[|\hat{\tau} - \tau| \geq \epsilon]\\
%             \intertext{Consider the following term, let $\delta = \epsilon/\tau$}
%             \textbf{Pr}[|\hat{\tau} - \tau| \geq \epsilon] &= \textbf{Pr}[|\hat{\tau} - \tau| \geq \delta \tau] = \textbf{Pr}[|\hat{\tau} - \tau|T \geq \delta T]\\ 
%             &\leq 2\exp \left(- \frac{\delta^2 T\tau}{3}\right)\\
%             &= 2\exp \left(- \frac{\epsilon^2 T\tau}{3\tau^2}\right)\\ 
%             &= 2\exp \left(- \frac{\epsilon^2 T}{3\tau}\right)\\
%         \end{align*}
%     \end{proofs}
% \end{frame}

% \begin{frame}{Proof Cont}
%     \setlength{\abovedisplayskip}{2pt}
%     \setlength{\belowdisplayskip}{2pt}
%     \setlength{\abovedisplayshortskip}{2pt}
%     \setlength{\belowdisplayshortskip}{2pt}
%     \begin{proof}[Minimum Sampling Size Proof Cont]
%         \begin{align*}
%             \intertext{Since we want $\delta$ to be the upper bound, we solve the following relation for $T$,}
%             \delta &\geq 2\exp \left(- \frac{\epsilon^2 T}{3\tau}\right)\\
%             T &\geq \frac{3 \tau}{\epsilon^2}\ln\left(\frac{2}{\delta}\right)\\
%         \end{align*}
%     \end{proof}

%     To implement this for real, we turn to data stream model proposed in \cite{Buriol_Frahling_Leonardi_Marchetti-Spaccamela_Sohler_2006}. The authors proposed the above algorithm along with multiple variants which approximate number of triangles with the same bounds. Moreover, they also proposed a one pass algorithm with expected update time $O(1 + s \cdot \log(|E|)/|E|)$ and same bounds.
% \end{frame}